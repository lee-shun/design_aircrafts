\chapter{导引弹道分析}
导弹的导引弹道形状完全取决于目标的运动状态、导弹的速度变化规律以及采用的导引规律。与导弹的动力学模型毫无关系。
最基本的目的是得到需用法向过载。
扩展过后的输入输出可以表示为:
$$
\left[
    \begin{matrix}
        \mbox{导弹的速度变化规律}\\
        \mbox{需用法向过载}\\
        \mbox{需用攻角}\\
        \mbox{需用弹体波束视线角}\\
        \mbox{需用舵偏角}
    \end{matrix}
    \right]
    =
    f\left(
        \left[
            \begin{matrix}
                \mbox{目标的运动}\\
                \mbox{动力系统参数}\\
                \mbox{导引规律}\\
                \mbox{升力阻力系数}\\
                \mbox{$M_z^{\alpha}$和$M_z^{\delta_z}$}
            \end{matrix}
            \right]
        \right)
$$
\section{导引弹道运动学}
导引规律对于导弹的约束主要有:
\begin{enumerate}
    \item 质心约束位置:三点法
    \item 速度方向的约束:速度追踪,比例导引
    \item 弹体姿态约束:弹体追踪法
\end{enumerate}
本书介绍的三点法;最终的需用法向过载由公式$f_y = V\dot{\theta}$得到。
\section{导引弹道动力学}
所谓导引弹道动力学的假设前提是弹体的姿态运动是没有过渡过程的,即执行机构与弹体
的姿态响应之间是一个纯比例环节,没有任何的延时。(也就是用的稳态方程建立的)
导弹的每一个过程都是平衡状态。(我们只关心质点动力学(包括位置矢量以及速度矢量))

{\kaishu 
关于方程组封闭的问题(铅锤平面内的质点动力学,力的方程建立在[弹道系或者速度系],
位置的运动学方程建立在地面系,见公式7-13):
因为弹体平衡攻角的加入,方程变得不封闭了。于是引入舵面控制量$\delta_z$和$\alpha$的关系,
前者由导引规律与控制规律共同给出。
但是实际上这个是控制率应该实现的任务,即"error$\rightarrow$command"
{\textcolor{red}{因而可以使用导引规律所给出的需用弹道倾角以及弹道倾角角速度作为输入,
给到原来的5个质心动力学方程组当中}}
}

需用攻角分析:根据需用攻角,可评价弹体动力学的线性化程度
需用攻角对线性化程度较好的导弹进行直接代数得出,不好的需要插值处理