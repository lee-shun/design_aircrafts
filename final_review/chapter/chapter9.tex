\chapter{滚转导弹动力学特性分析}
{\kaishu $^*$关于滚转导弹动力学建模问题的几点说明:
\begin{enumerate}[i]
\item 重新建立的两个坐标系,准速度、准弹体坐标系。
\item 准速度与准弹体的夹角定义为:$\alpha^{*},\beta^{*}$
\item 控制力与控制力拒由于弹体的"低通滤波"特性的存在,只响应周期平均控制力以及控制力矩
\item 控制力以及控制力矩的大小由产生的控制力的大小决定,方向由“弹体”和控制力变化的周期的初相位决定
\item 控制力系数$F_{y4},F_{x4}$实际上表示了控制力的方向($sin\varphi,cos\varphi,\varphi\mbox{是初相位}$)
\item 质心动力学还是建立在弹道坐标系上,转动动力学写在准弹体坐标系上
\end{enumerate}
}
\section{滚转弹的状态空间、传函以及复分析方法}
{\heiti 状态空间:}

忽略了重力的影响之后,滚转弹的状态空间方程是一个4阶的被控制系统,状态向量为:
$\left[
\begin{smallmatrix}
    \Delta\theta\\
    \Delta\alpha\\
    \Delta\dot{\psi}\\
    \Delta\beta
\end{smallmatrix}
\right]
$
控制量为控制力系数。

{\heiti 复分析方法:}

由于控制力系数,气动角以及姿态角(当然没有滚转角)是成对出现的,引入复控制力,复气动角和复姿态角的概念。
引入上述方法之后,4阶系统在形式上变为了2阶,但是状态向量以及控制输入都变成了复数。(教材公式9-9)

{\heiti 传函:}

由由复分析方法得到的“2阶”状态空间方程可以得到复数传递函数,分母为2阶,若转换为分散的两个传递函数,则变为4阶(形式改变,物理含义未变)
\section{滚转弹动态稳定性以及失稳分析}
\subsection{稳定分析}
{\heiti 稳定性的条件:}

由复分析方法得到的“2阶”状态空间方程以及传递函数,直接由劳斯稳定性判据可以得到稳定的条件。
最终的结果见教材公式9-38,实际上是滚转弹动力系数的组合判据,即使导弹是静稳定的,滚转条件下也不一定稳定。

{\heiti 稳定转速边界:}

根据第8章的倾斜稳定的导弹的固有频率$\omega_m$表达式结合稳定性条件,可得出滚转弹的滚转角速度的上限,与$\omega_m$正相关。
\subsection{失稳分析}
\subsubsection{马格努斯效应}
旋转导弹在一定的攻角飞行时,在侧向方向产生马格努斯力和力矩,这是由于攻角产生侧向的力和力矩的,因而存在着交联耦合现象。
{\kaishu 即使定性的分析之下,马格努斯效应经过侧侧向力$\rightarrow$侧滑角$\rightarrow$纵向力,是负反馈的,但是这个减少也可能是不稳定的}
\subsubsection{失稳机理}
由于陀螺力矩对于失稳的影响小于10\%,因而可以忽略不予考虑
由滚转导弹小扰动方程出发,可以绘制多输入条件下的方框图。最终得到关于两个控制为输入的方框图,单就纵向通道来看,
从$A_{K_y}\rightarrow\alpha$来看,这是一个高阶系统,虽然有负反馈的存在,当开环增益增大到一定程度时会发生
动态失稳。

{\kaishu 从bode图的角度分析,高频段的相角滞后将达到-360°,因而当系统的截止频率,
即开环增益达到一定程度,系统就会不稳定,就纵向通道来说,静稳定度增大,截止频率降低,系统的开环增益降低,
稳定性提高转速增加,开环增益加大,稳定性下降。

\textcolor{blue}{算例:}

实际上是利用闭环的手段(劳斯判据),开环(频域)手段判断系统的稳定性以及稳定性边界。}
\subsection{滚转弹转速设计}
\subsubsection{转速设计约束条件}
\begin{enumerate}
    \item 弹体动力学
    \item 舵机动力学
    
    {\kaishu 舵机的响应频率以及延时的约束,转速越快,影响越大}
    \item 控制系统设计
    
    {\kaishu 控制信号的频率大于2-2.5倍的误差信号频率,误差信号取决于制导回路的带宽,
    他决定了导弹转速设计的下边界}
    \item 结构动力学
    
    {\kaishu 结构谐振破坏}
    \item 发动机内弹道稳定性
\end{enumerate}
\subsubsection{弹体动力学对转速的约束}
必须使控制力的频率高于弹体固有频率的3-5倍,保证弹体的滤波效果,即下边界收到弹体固有频率的限制,
上边界则受到弹体动力学稳定性边界的约束。
\subsubsection{鸭式单通道滚转弹最优转速设计}
鸭式导弹的存在最优转速(正常式不存在,忽略了直联项之后也不存在。。。):

由气动力产生的那部分加速度在滞后并且大小逐渐变小。由直联项产生的加速度恒定不变,{\kaishu \textcolor{blue}
{当滞后角几乎接近180°时,恰恰会出先2种加速度幅值一致的情况,此时的瞬时加速度几乎为零,但是周期平均加速度不会。
此时弹体滤波滤的“最狠”,瞬时的控制力的扰动最小,就会出现所谓最优频率以及最优转速}}