\chapter{战术导弹总体方案设计步骤以及数据流管理}
\section{战术导弹总体方案设计特点}
\begin{enumerate}[1.]
    \item 涉及学科专业多
    \item 涉及参数多
    \item 设计过程复杂
    \item 具有经验继承性
\end{enumerate}
\section{战术导弹总体方案设计}
\begin{enumerate}[1.]
\item 总体概要设计
    \begin{enumerate}[i]
        \item 确定弹身直径
        \item 确定战斗部类型
        \item 确定导弹速度方案
        {\kaishu 三种基本形式:
            \begin{enumerate}[a]
                \item 加速助推+无动力飞行
                \item 加速助推+等速续航
                \item 加速助推+加速续航+无动力飞行
            \end{enumerate}
        }
        \item 确定弹道方案
        \item 确定基本的动力系统形式
        \item 确定制导体制
        \item 确定导引规律
        \item 确定对发动机推进剂的性能要求
        \item 确定导弹的基本气动布局形式
        \item 确定导弹的基本控制模式
    \end{enumerate}
\item 战斗部方案设计
    三点说明:
    \begin{enumerate}[i]
        \item a
    \end{enumerate}
\item 起飞质量设计
    三点说明:
\begin{enumerate}[i]
    \item a
\end{enumerate}
\item 发动机方案设计
\item 导引弹道运动学分析
\item 控制系统概要设计
\item 第一轮结构设计
\item 第一轮气动设计
\item 第二轮结构设计
\item 第二轮气动设计
\item 导引弹道动力学分析
\item 弹体动态特性分析
\item 制导回路设计
\item 自动驾驶仪设计
\item 有控刚体弹道仿真

\end{enumerate}

