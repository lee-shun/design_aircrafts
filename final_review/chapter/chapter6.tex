\chapter{导弹部位安排设计}
\section{导弹部位安排的基本要求}
\begin{enumerate}
    \item 质心位置,转动惯量
    \item 各个系统及密结合
    \item 结构紧凑
    \item 安装合理,工艺性好
    \item 维护性和存储
\end{enumerate}
\section{导弹的部位安排设计}
\subsubsection*{静稳定度}
焦点以及质心均会在导弹飞行过程中变化,其中焦点最常见的是前移
\subsubsection*{固有频率}
利用飞行力学中$\omega_m = \sqrt{a_{24}+a_{22}a_{34}}\approx\sqrt{a_{24}} = \sqrt{\frac{-m_z^{\alpha}qSL}{J_z}}$
可以得到焦点位置的下限值。
\subsubsection*{机动性}
静稳定度越大,机动性能越差(机动性能由单位舵偏角过载$\frac{n_y}{\delta_z}$表示)
