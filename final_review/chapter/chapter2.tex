\chapter{战术导弹战术技术要求分析}
\section{战术技术要求的内容}
\begin{enumerate}[1.]
    \item 目标的典型特征
    
    {\kaishu 利用目标的运动,几何,辐射等,确定自身的引信,战斗部,速度,过载等}
    \item 射程指标
    
    {\kaishu 定义了有效射程}
    \item 作战高度范围
    
    {\kaishu 气压高度指示绝对高度,无线电测高反应相对高度,高度的变化会引起增益以及固有频率的下降}
    \item 环境温度范围
    
    {\kaishu 元器件的工作稳定性以及精度;分为工作环境温度,存储环境温度,温度冲击}
    \item 导弹速度特性或最大攻击时间
    
    {\kaishu 速度特性定义:速度随时间变化的曲线,以及速度特征量,例如:最大,平均速度,加速度,速度比等;在确定下导弹的速度特性之后,
    导弹的飞行速度范围,飞行时间,射程,高度等参数均可确定,进而进行导弹的外形设计,质量估算,并确定发动机的推力特性}
    \item 导弹级数
    
    {\kaishu 导弹可以在飞行过程中获得良好的加速性能,有利于导弹的飞行性能优化,提高}
    \item 导弹质量、几何尺寸限制
    \item 制导体制
    \item 命中精度
    
    {\kaishu 圆概率偏差,命中概率}
    \item 战斗部威力
    
    {\kaishu 常用破片的的杀伤半径,衡量其威力大小的指标}
    \item 发射方式
    \item 突防能力、战场生存能力
    
    {\kaishu 突防能力:飞跃敌方的防御之后仍然可以保持初级功能}
    \item 抗干扰能力
    
    {\kaishu 来自环境和敌方的强烈干扰}
    \item 武器系统可靠性
    \item 经济性
\end{enumerate}