\chapter{绪论}
\section{作为设计对象的战术导弹}

导弹可以分为战术导弹以及战略导弹,战术导弹是攻击军事型目标的导弹。

导弹武器系统的组成:导弹系统、火控系统和技术保障系统。
\subsection{战术导弹的特点}
\begin{enumerate}[1)]
    \item 命中精度高
    
    {\kaishu 在$3000\sim5000m$的最大射程上,圆概率偏差不超过0.5m}
    \item 机动能力强
    
    {\kaishu 目前的空空导弹以及地空导弹可以提供$25\sim30g$法向过载,先进导弹可提供$50\sim70g$}
    \item 系统组成以及结构复杂
    \item 大量采用高新技术
    \item 品种多、产量大、更新换代快
\end{enumerate}
\subsection{战术导弹研制过程}
\begin{enumerate}[1)]
    \item 指标可行性论证阶段
    \item 方案设计阶段
    \item 初样阶段
    \item 试样阶段
    \item 设计定型阶段
    \item 生产定型阶段
\end{enumerate}
\subsection{导弹总体设计}
\textbf{静稳定度对于各个分系统的影响:}
\begin{enumerate}[1)]
    \item 发动机:燃料消耗导致的质心移动、推力偏心
    \item 控制系统:静稳定度越大,增益以及阻尼越小,固有频率越高(有的导弹要求增益大,
    频率高,应该用自驾仪提升内回路频率)
    \item 结构:改变质心位置
    \item 舵机:在平衡状态下,舵机产生的控制力拒应该与弹体的恢复力矩相等
    $\left|M^{\alpha}_z\right|\alpha_B = \left|M^{\delta_z}_z\right|\delta_z$ 
    即,在同样的攻角条件下,静稳定度越大,所需舵偏角越大(就是前面增益问题的体现)
    \item 气动:焦点$\rightarrow$静稳定度
    \item 发射:滚转弹获得初始转速(膛线),初始转速的设计与静稳定度有关。
\end{enumerate}
\subsection{总体设计工作内容}
\begin{enumerate}[1)]
    \item 新型战术导弹概念研究
    \item 武器系统以及导弹系统总体方案论证
    \item 导弹总体参数设计、优化、选择
    \item 导弹总体布局设计
    \item 飞行弹道以及制导规律选择
    \item 弹体动力学特性分析
    \item 闭环有控弹道仿真
    \item 弹上各个系统设计方案
    \item 环境工程以及电磁兼容总体设计
    \item 可靠性以及维修性设计
    \item 导弹试验设计
    \item 总体性能综合评估
\end{enumerate}
