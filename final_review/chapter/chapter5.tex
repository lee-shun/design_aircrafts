\chapter{导弹外形设计}
\section{总体设计与气动外形设计的关系}
\subsection{外形与机动性}
导弹的可用过载由最大升力以及重力决定,最大升力由公式$Y_{max}=c_y^{\alpha}qS\alpha_k$确定;
而$c_y^{\alpha}$是导弹几何外形以及马赫数的函数,因而几何外形可以决定机动性。
\subsection{外形与起始质量}
起始质量设计时要用到折算阻力系数(这个值在前一步是工程师假设的),此系数的是气动外形参数(例如翼型,弹翼几何尺寸等)的函数。
\section{气动外形设计的主要要求}
\begin{enumerate}
    \item 气动特性
    \item 机动性
    \item 稳定性
    \item 操纵性
    \item 其他(部位安排,发射方式,制导要求等)
\end{enumerate}
\section{气动布局形式}
\subsection{按周向配置形式分类}
\subsubsection*{平面式(“飞机式”方案)}
\begin{enumerate}
    \item 阻力小、质量轻,远程作战
    \item BTT转弯模式下,升力对准目标
    \item 结构紧凑,悬挂方便
    \item 两种转弯模式,STT\&BTT:{\kaishu 平面(侧滑)转弯,侧滑角产生侧向力,本方式的侧向过载较小;
    协调(倾斜)转弯:滚转后使升力对准所需机动方向,获得的侧向过载大,但是过渡时间长,对付机动目标困难}
\end{enumerate}
\subsubsection*{空间配置方案}
\begin{enumerate}
    \item “十”或“$\times$”形式,
    \item 导弹在俯仰上的要求大于偏航方向,采用“H”或者“$\times$”形式比较合适。
\end{enumerate}
\subsection{按纵向配置形式分类}
\subsubsection*{正常式}
负升力、响应特性(较)慢
气动耦合小、舵面受载小,
栅格翼(跨声速阻力大,控制效率低,雷达截面大)
\subsubsection*{鸭式}
响应特性快、舵机安排方便、
失速点低、洗流干扰$\rightarrow$“分离式鸭式布局”,鸭舵前加装固定翼面,提高失速点,提高控制效率。
\subsubsection*{旋转弹翼式}
响应特别快,缺点也很明显,铰链力矩大等问题
\subsubsection*{无尾式}
弹翼的移动对于弹的稳定性以及操纵效率的影响很大(十分敏感)
\section{外形几何参数选择}
\subsection{弹翼设计}
\subsubsection{弹翼设计基本问题}
\begin{enumerate}
    \item 良好的气动特性,机动飞行时升阻比最大,焦点的变化小
    \item 静稳定度合适
    \item 刚度强度下,质量小、工艺性好
    \item 结构紧凑
    \item 部位安排方便
\end{enumerate}
\subsubsection{几何形状的影响}
\begin{enumerate}
    \item 展弦比:
    
    {\kaishu 
    \begin{enumerate}[i]
        \item 展弦比增加,升力系数增加,但是增加的速度减慢,$M_a$大时更加明显
        \item 展弦比大,升阻比变大,但会受到$M_a$影响
        \item 展弦比增大,临界攻角降低
        \item 展弦比增大,结构刚度变差
        \item [*] 因而亚音速弹用大展弦比,超音速弹反之。高速防空导弹小于2
    \end{enumerate}
    }
    \item 根梢比
    
    {\kaishu 
    \begin{enumerate}[i]
        \item 超音速条件下,三角翼更加优越:焦点位置移动更小,升阻比大(尤其在小展弦比下)
        \item 三角翼的压心靠近翼根;
    \end{enumerate}
    }
    \item 后掠角(前缘)
    
    {\kaishu 
    \begin{enumerate}[i]
        \item 后掠角和马赫锥角的大小判断是否是超音速前缘
        \item 亚音速时,后掠角大,升力系数降低,超音速不明显;
        \item 超音速波阻随后掠角增大而降低
    \end{enumerate}

    结论:
    \begin{enumerate}[i]
        \item 主航段为超音速,则需设计成有后掠
        \item 亚音速时,提高升力系数,选用较小的后掠角,应大于45°
        \item 为了减小波阻,最大厚度线后掠角不应与马赫线重合
    \end{enumerate}
    }
    \item 翼型
    
    \begin{enumerate}[i]
        \item 菱形翼的减轻波阻的效果最好,但是考虑到加工以及结构上的要求,会选择其他翼型。
        \item 减小厚度可以减小波阻。
    \end{enumerate}
\end{enumerate}
\subsubsection{弹翼面积}
机动性(升力)以及射程(阻力)的要求。可根据:
\begin{enumerate}
    \item 机动性设计弹翼
    \item 可用攻角以及机动性要求设计弹翼面积。
    \item 可用攻角以及展弦比设计弹翼
\end{enumerate}
\subsection{舵面的尺寸}
注意:

\begin{enumerate}
    \item 控制效率
    \item 铰链力矩特性
\end{enumerate}
主翼控制的导弹:

\begin{enumerate}
    \item 升力的作用点(附:升力增量--焦点)在导弹质心之后:
    
    此时,对于静稳定的导弹,$m_z^{\alpha}$ 、
    $m_z^{\delta_z}$ 都是小于零的,此时,攻角产生的升力与弹翼偏转产生的升力方向相反,升力有所损失。
    \item  重合
    
    弹翼偏转时,弹身的气动特性没有充分的利用。(弹翼偏转对于升力的贡献是0)
    \item 之前
    
    与情况一相反,有利
\end{enumerate}

设计准则:

\begin{itemize}
    \item 舵偏角“够用”即可
    \item 舵面尺寸不宜过大,防止过载
\end{itemize}

\subsection{弹身几何参数}
头部:

\begin{itemize}
    \item 弹身头部外形设计:圆锥、抛物线、尖拱形、半球形
    \item 考虑波阻以及气动加热问题(导引头 )、结构强度等因素。
\end{itemize}

尾部:

尾部长细比,收缩比,影响到尾部的摩擦阻力以及底部的(压差)阻力

中段:

长细比影响到零升阻力

弹身直径由战斗部,发动机以及外界限制确定。






